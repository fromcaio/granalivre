\chapter{Conclusão}\label{cap:conclusao}

Este documento teve como objetivo registrar a estruturação inicial do projeto, contemplando desde a contextualização e fundamentação teórica até a definição metodológica e os artefatos desenvolvidos na primeira \textit{sprint}. A documentação serviu como guia para alinhar a equipe quanto às tecnologias utilizadas, à organização do repositório e à definição dos papéis e responsabilidades.

Os diagramas de casos de uso, os protótipos de baixa fidelidade e a modelagem relacional do banco de dados forneceram uma visão clara e compartilhada do sistema, estabelecendo uma base sólida para o desenvolvimento a ser realizado nas próximas etapas. Além disso, a elaboração do cronograma permitiu definir marcos objetivos e mensuráveis, garantindo maior controle sobre o avanço do projeto.

Espera-se que esta documentação sirva não apenas como registro do processo de idealização do sistema, mas também como referência para o acompanhamento e a evolução do desenvolvimento. Nas próximas \textit{sprints}, a equipe avançará na implementação das funcionalidades planejadas, consolidando a visão apresentada neste trabalho inicial.