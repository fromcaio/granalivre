\chapter{Conclusão}\label{cap:conclusao}

Durante o desenvolvimento desde trabalho, foi possível constatar a complexidade e a extensão do processo semestral realizado pelas coordenações de curso. Nesse contexto, a implementação de sistemas informatizados assume um papel fundamental na modernização e otimização das atividades gerenciais, contribuindo para a melhoria da comunicação interna e externa entre os setores, a agilidade e padronização das tarefas, além de minimizar a ocorrência de retrabalhos.

\section{Contribuições}

O trabalho desenvolvido resultou em uma ferramenta que centraliza, de forma eficiente, o processo de registro de disciplinas e a alocação de encargos didáticos em um único sistema. Com isso, oferece suporte às coordenações de curso e departamentos na organização e no planejamento dos encargos didáticos a cada período letivo. Além disso, simplifica o processo de definição das disciplinas a serem ofertadas em cada período, proporcionando aos departamentos uma visão clara e detalhada de suas responsabilidades. Por fim, o sistema busca agilizar e otimizar a indicação de docentes para os encargos atribuídos, tornando esse processo mais eficiente e assertivo.


\section{Limitações}\label{sec:limitacoes}

Embora o trabalho apresentado tenha alcançado os objetivos propostos, durante o processo de desenvolvimento foram identificadas possíveis limitações em seu funcionamento. O estudo foi conduzido com foco específico na aplicação ao curso de Ciência da Computação da Universidade Federal de São João del Rei, o que pode resultar em restrições caso seja adaptado para outros cursos ou instituições. Contudo, o sistema foi projetado de maneira a ser flexível, permitindo futuras implementações e customizações relacionadas às suas regras de negócio.


\section{Trabalhos Futuros}

Por fim, com o objetivo de fomentar a continuidade deste estudo, foram sugeridas linhas de trabalho futuro, identificadas a partir das necessidades observadas durante a aplicação do sistema em situações reais.

Conforme apresentado na Sessão \ref{sec:limitacoes}, a possibilidade de novas implementações, considerando as demandas de outros cursos e instituições, constitui uma oportunidade para ampliar a escala e a abrangência do sistema.

Adicionalmente, a inclusão de módulos destinados ao gerenciamento de atividades como a elaboração de horários, considerando a disponibilidade de professores, e o planejamento do uso de salas e laboratórios, representa outro conjunto de funcionalidades semestrais que poderiam enriquecer o sistema. Tais melhorias, especialmente com a aplicação de técnicas de Inteligência Artificial, têm o potencial de tornar o software ainda mais robusto e completo.