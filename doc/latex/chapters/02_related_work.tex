\chapter{Fundamentação Teórica}\label{cap:fund_teorica}

\section{Conceitos Gerais}

O desenvolvimento de sistemas web modernos envolve a integração de múltiplas camadas e tecnologias, que vão desde a interface do usuário até a lógica de negócio e a persistência de dados. O projeto \textit{GranaLivre} adota uma arquitetura baseada na separação entre \textit{front-end} e \textit{back-end}, de modo a facilitar a escalabilidade, a colaboração entre desenvolvedores e a manutenção do sistema.

\section{Tecnologias Utilizadas}

\subsection{HTML, CSS e JavaScript}
A base do desenvolvimento \textit{front-end} é composta por HTML, CSS e JavaScript, tecnologias fundamentais para a construção de páginas web.  
O HTML organiza o conteúdo e define a estrutura semântica da aplicação, enquanto o CSS gerencia a aparência visual. Já o JavaScript possibilita a interação dinâmica e o controle dos componentes da interface, tornando a aplicação mais responsiva e intuitiva para o usuário.

\subsection{Next.js}
O \textit{Next.js} é um framework baseado em React que fornece uma estrutura organizada para o desenvolvimento de interfaces modernas.  
A escolha do Next.js justifica-se pela clareza no desenvolvimento, pela facilidade de criação de componentes reutilizáveis e pela boa integração com o ecossistema JavaScript.

\subsection{Node.js}
O \textit{Node.js} é utilizado como ambiente de execução do Next.js, permitindo que o código JavaScript seja processado no lado do servidor quando necessário. Essa tecnologia é fundamental para lidar com operações assíncronas e de alta performance, garantindo maior flexibilidade ao front-end.

\subsection{Python e Django}
O \textit{back-end} do sistema é desenvolvido em Python, utilizando o framework \textit{Django}. Esse framework fornece uma estrutura robusta para o desenvolvimento rápido de aplicações web, oferecendo recursos nativos como sistema de autenticação, gerenciamento de banco de dados via ORM e uma arquitetura organizada em padrões de projeto.  
O Django foi escolhido por sua maturidade, comunidade ativa e facilidade de integração com diferentes bancos de dados relacionais.

\subsection{SQLite}
O banco de dados escolhido para o projeto é o \textit{SQLite}, um sistema de gerenciamento de banco de dados relacional (RDBMS) \textit{serverless} (sem servidor). Sua principal vantagem reside na leveza e simplicidade, armazenando todo o banco de dados em um único arquivo no sistema de arquivos. Esta escolha é estratégica para o \textit{GranaLivre} por diversos motivos:
\begin{itemize}
    \item \textbf{Execução Local e Portabilidade:} Por não requerer um processo de servidor separado, o SQLite é ideal para aplicações onde o usuário executa o software localmente, eliminando a complexidade de gerenciar uma instância de banco de dados externa. Isso também facilita a implementação de futuras versões desktop, onde a base de dados pode ser facilmente empacotada com a aplicação.
    \item \textbf{Facilidade na Geração de Instaladores:} Ao empacotar a aplicação, o arquivo do banco de dados SQLite pode ser incluído diretamente, resultando em instaladores mais leves e fáceis de distribuir, sem a necessidade de configurar um banco de dados complexo no ambiente do usuário.
    \item \textbf{Desenvolvimento Simplificado:} A integração com o ORM do Django é transparente e a manutenção da camada de persistência de dados se torna mais direta, pois não há dependências de serviços de banco de dados externos durante o desenvolvimento.
    \item \textbf{Desempenho para Aplicações de Usuário Único:} Para o perfil de uso do \textit{GranaLivre}, que foca no controle financeiro pessoal e geralmente opera em modo de usuário único, o desempenho do SQLite é mais do que adequado e eficiente.
\end{itemize}
A compatibilidade nativa do Django com SQLite facilita seu uso e manutenção, tornando-o uma escolha robusta para as necessidades atuais e futuras do projeto.

\subsection{Docker}
Para simplificar o processo de desenvolvimento, implantação e utilização do sistema, será utilizado o \textit{Docker}. Essa tecnologia permite a criação de ambientes isolados e padronizados, garantindo que a aplicação possa ser executada de maneira consistente em diferentes máquinas e servidores.  
Com isso, tanto desenvolvedores quanto usuários finais terão maior facilidade em instalar e utilizar o \textit{GranaLivre}.

\subsection{Git e GitHub}
O Git é um sistema de controle de versão distribuído, criado para gerenciar o histórico de alterações em projetos de software. Ele permite que múltiplos desenvolvedores trabalhem de forma colaborativa, rastreando cada modificação no código-fonte, facilitando a reversão de erros e permitindo a criação de ramificações (\textit{branches}) para o desenvolvimento de novas funcionalidades sem afetar a versão principal do projeto.

O GitHub, por sua vez, é uma plataforma de hospedagem de código-fonte que utiliza o Git como seu principal sistema de controle de versão. Ele funciona não apenas como um repositório remoto, mas como um centro de colaboração para projetos de software, especialmente os de código aberto.

No contexto do projeto \textit{GranaLivre}, o uso de Git e GitHub é fundamental por diversas razões:
\begin{itemize}
\item \textbf{Controle de Versão:} Todo o histórico de desenvolvimento do projeto será rastreado, garantindo segurança e organização.
\item \textbf{Colaboração Aberta:} O GitHub permite que qualquer desenvolvedor da comunidade possa "forkar" (criar uma cópia) o repositório, implementar melhorias e submeter suas contribuições através de \textit{Pull Requests}, que serão revisadas antes de serem integradas ao código principal.
\item \textbf{Gerenciamento do Projeto:} A plataforma oferece ferramentas como \textit{Issues} (para relatar bugs e sugerir novas funcionalidades) e \textit{Projects} (para organizar o fluxo de trabalho), centralizando a comunicação e o planejamento das tarefas.
\item \textbf{Divulgação e Transparência:} Manter o código-fonte em um repositório público no GitHub serve como o principal "cartão de visita" do projeto, atraindo novos colaboradores e garantindo total transparência sobre o funcionamento do sistema. O repositório do \textit{GranaLivre} pode ser acessado em: \url{\urlrepositorio}.
\end{itemize}
Dessa forma, o GitHub não é apenas uma ferramenta de desenvolvimento, mas também o principal ponto de encontro da comunidade em torno do \textit{GranaLivre}.

\section{Arquitetura Geral}
O sistema segue um modelo baseado na separação entre \textit{front-end} e \textit{back-end}. O \textit{front-end}, desenvolvido com Next.js, é responsável pela interface do usuário e comunicação com a API. Já o \textit{back-end}, construído em Django, expõe serviços e gerencia a persistência dos dados em PostgreSQL.  
O uso do Docker garante que toda essa arquitetura seja facilmente replicável e portável, favorecendo tanto o desenvolvimento colaborativo quanto a implantação.

\section{Trabalhos Relacionados}

Existem diversas aplicações de controle financeiro já disponíveis, tanto em formato proprietário quanto em projetos de código aberto. No entanto, muitas dessas soluções encontram-se em estágios avançados de desenvolvimento, com arquiteturas complexas e documentação predominantemente em inglês, o que pode dificultar a participação de novos colaboradores — especialmente estudantes e desenvolvedores brasileiros em formação.

O diferencial do \textit{GranaLivre} não está apenas em sua proposta funcional, mas sobretudo em seu caráter formativo e comunitário. O sistema foi concebido como uma aplicação em estágio inicial, organizada desde o início com boas práticas de desenvolvimento e documentação em português. Dessa forma, busca-se criar um ambiente convidativo para que desenvolvedores brasileiros possam contribuir ativamente, aprendendo e crescendo junto ao projeto.

A intenção é, portanto, não apenas oferecer uma alternativa de software de finanças pessoais, mas também construir um espaço colaborativo que valorize a cultura local, promova a troca de conhecimento e incentive a participação da comunidade no ecossistema de software livre.