\chapter{Introdução}\label{cap:intro}

\section{Contextualização}
O controle financeiro pessoal é uma prática essencial para o equilíbrio das finanças de qualquer indivíduo. Apesar disso, muitas pessoas ainda encontram dificuldades em organizar receitas, despesas e investimentos de forma clara e acessível. Nesse cenário, surgem ferramentas digitais que auxiliam na gestão financeira, mas grande parte delas é paga, limitada ou não oferece transparência no uso dos dados.

Com base nessa necessidade, o projeto \textit{GranaLivre} foi idealizado como uma plataforma \textit{open source} voltada para o gerenciamento de finanças pessoais. Seu propósito é oferecer uma solução simples, intuitiva e acessível, permitindo que qualquer usuário tenha maior controle sobre suas finanças sem depender de sistemas proprietários.

\section{Motivação}
A motivação para o desenvolvimento do \textit{GranaLivre} está diretamente ligada à importância da autonomia financeira e à democratização do acesso a ferramentas de planejamento. Além disso, muitas soluções existentes são documentadas apenas em inglês, o que dificulta a participação de desenvolvedores brasileiros. O \textit{GranaLivre}, ao disponibilizar toda a documentação em português, busca valorizar a comunidade local e ampliar as oportunidades de contribuição, considerando a cultura e as necessidades específicas do público brasileiro.

\section{Objetivos}

\subsection{Objetivo Geral}
Desenvolver uma aplicação web de finanças pessoais que permita aos usuários organizar receitas, despesas, investimentos e patrimônio de forma simples e intuitiva.

\subsection{Objetivos Específicos}
\begin{itemize}
    \item Disponibilizar uma plataforma \textit{open source}, gratuita e acessível a qualquer usuário;
    \item Permitir o registro e acompanhamento de receitas e despesas;
    \item Facilitar a gestão de contas recorrentes, assinaturas e metas financeiras;
    \item Oferecer visualização clara do saldo disponível e do patrimônio acumulado;
    \item Produzir documentação em português para engajar a comunidade de desenvolvedores brasileiros;
    \item Estabelecer uma base sólida para futuras evoluções do sistema, integrando recursos mais avançados de planejamento financeiro.
\end{itemize}

\section{Justificativa}
A justificativa do \textit{GranaLivre} está na necessidade crescente de ferramentas confiáveis, acessíveis e abertas para controle financeiro. Ao adotar o modelo \textit{open source}, o projeto garante não apenas a transparência no uso dos dados, mas também a possibilidade de evolução contínua por meio da colaboração da comunidade. Outro diferencial está na produção de documentação totalmente em português, o que reduz barreiras de entrada e valoriza a cultura local, permitindo que desenvolvedores brasileiros tenham maior facilidade para participar ativamente da evolução da plataforma. Assim, busca-se construir um sistema que ajude os usuários a alcançar maior organização e independência financeira.