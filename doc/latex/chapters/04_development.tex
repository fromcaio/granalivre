\chapter{Desenvolvimento}\label{cap:desenvolvimento}

Este capítulo apresenta os artefatos produzidos durante a primeira \textit{sprint} do projeto, contemplando a modelagem inicial do sistema. Os elementos aqui descritos foram fundamentais para alinhar a visão da equipe em relação às funcionalidades, à interface do usuário e à estrutura do banco de dados.

\section{Diagrama de Casos de Uso}
O diagrama de casos de uso tem como objetivo ilustrar, de forma abstrata, as principais funcionalidades do sistema e os atores envolvidos. Este artefato permitiu identificar, já na fase inicial, quais interações devem ser suportadas pela aplicação.

\begin{figure}[H]
    \centering
    \includesvg[width=1\textwidth]{imgs/diagrama-casos-de-uso.svg}
    \caption{Diagrama de Casos de Uso do Sistema}
    \label{fig:casos_uso}
\end{figure}

\subsection{Descrição dos Casos de Uso}
\begin{itemize}
    \item \textbf{Entrar}:Permite que o usuário acesse o sistema utilizando suas credenciais (e-mail e senha).
    \item \textbf{Cadastrar}: Possibilita que novos usuários criem uma conta no sistema, informando seus dados básicos.
    \item \textbf{Visualizar Resumo Financeiro}: Exibe uma visão geral da situação financeira do usuário, incluindo saldo disponível, movimentações recentes, contas a pagar, patrimônio. Também mostra um gráfico que relaciona os bens e as despesas, e ainda permite ao usuário registrar despesas e entradas eventuais de maneira rápida.
    \item \textbf{Gerenciar Saídas}: Permite registrar, editar e excluir despesas. Inclui a categorização de gastos para facilitar a análise.
    \item \textbf{Gerenciar Contas Recorrentes}: Facilita o controle de despesas fixas, como contas de serviços públicos, aluguel, entre outras. Permite adicionar, editar e remover contas recorrentes.
    \item \textbf{Gerenciar Entradas}: Permite registrar, editar e excluir receitas. Inclui a categorização de entradas para facilitar a análise.
    \item \textbf{Gerenciar Patrimônio}: Possibilita o cadastro de bens, como imóveis, veículos ou outros ativos de valor, permitindo acompanhar sua valorização ou depreciação.
    \item \textbf{Gerenciar Investimentos}: Permite cadastrar, acompanhar e atualizar informações sobre investimentos. Inclui também a possibilidade de liquidar investimentos, registrando sua venda ou encerramento.
    \item \textbf{Gerenciar Conta}: Permite ao usuário atualizar suas informações pessoais, alterar a senha e excluir sua conta do sistema.
    \item \textbf{Sair}: Permite que o usuário encerre sua sessão no sistema de forma segura.
\end{itemize}

\section{Protótipos de Baixa Fidelidade}
Os protótipos de baixa fidelidade foram desenvolvidos para representar a organização e o fluxo das telas do sistema, servindo como base para discussões e refinamentos futuros. Essa etapa auxilia na antecipação de problemas de usabilidade e na validação das ideias de interface junto à equipe.

% \begin{figure}[H]
%     \centering
%     \includegraphics[width=0.9\textwidth]{imagens/prototipo1.png}
%     \caption{Protótipo de baixa fidelidade da tela de login}
%     \label{fig:prot_login}
% \end{figure}

% \begin{figure}[H]
%     \centering
%     \includegraphics[width=0.9\textwidth]{imagens/prototipo2.png}
%     \caption{Protótipo de baixa fidelidade da tela inicial}
%     \label{fig:prot_inicial}
% \end{figure}

% Adicionar outras telas conforme necessário

\section{Diagrama Relacional do Banco de Dados}
A modelagem do banco de dados foi realizada com o objetivo de estruturar as entidades fundamentais para o funcionamento do sistema, assegurando integridade referencial e eficiência nas consultas. O diagrama relacional reflete as tabelas e os relacionamentos definidos.

% \begin{figure}[H]
%     \centering
%     \includegraphics[width=0.9\textwidth]{imagens/der.png}
%     \caption{Diagrama Relacional do Banco de Dados}
%     \label{fig:der}
% \end{figure}