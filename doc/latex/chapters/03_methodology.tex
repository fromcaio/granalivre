\chapter{Metodologia}\label{cap:metodologia}

Para alcançar o objetivo de desenvolver uma plataforma de publicação robusta, escalável e intuitiva, a construção do sistema FromCaio seguiu um processo de desenvolvimento iterativo e incremental. Esta abordagem permitiu a entrega de valor contínua e a adaptação a novos requisitos de forma flexível. O projeto foi estruturado nas seguintes macroetapas:

\begin{enumerate}
    \item Pesquisa e Definição das Tecnologias, cujos fundamentos são detalhados no Capítulo \ref{cap:ref_teorico}.
    \item Levantamento e Análise de Requisitos.
    \item Projeto da Arquitetura e Modelagem de Dados.
    \item Implementação e Validação do Sistema.
\end{enumerate}


\section{Levantamento e Análise de Requisitos}\label{section:req}

Nesta fase inicial, a coleta de requisitos foi realizada por meio de uma abordagem multifacetada. Primeiramente, conduziu-se uma análise de plataformas de publicação de conteúdo já consolidadas (como Medium, DEV.to e blogs acadêmicos) para identificar funcionalidades essenciais, padrões de interface e lacunas que o FromCaio poderia suprir. Em paralelo, foram realizadas discussões com o Professor Orientador para alinhar os objetivos do projeto com as necessidades de educadores e produtores de conteúdo técnico. A partir dessa análise, foram definidos os requisitos funcionais, como o sistema de autenticação, o editor baseado em Markdown, o suporte a LaTeX e o realce de sintaxe de código, e os requisitos não funcionais, como desempenho, usabilidade e portabilidade do conteúdo.


\section{Projeto da Arquitetura e Modelagem de Dados}

Com os requisitos definidos, o projeto focou na definição de uma arquitetura limpa e desacoplada, visando a manutenibilidade e a escalabilidade futuras da plataforma.

A arquitetura do sistema foi dividida em duas partes principais: o \textit{back-end} (servidor e API) e o \textit{front-end} (interface do usuário). Para o \textbf{back-end}, optou-se pelo uso de \textbf{Node.js} com o framework \textbf{Express.js} para a construção de uma API RESTful. Essa escolha se deve à performance suficiente em operações de entrada/saída (I/O) e ao vasto ecossistema de bibliotecas, que facilitam a implementação de funcionalidades como autenticação, processamento de Markdown e interação com o banco de dados. A lógica de negócio foi isolada em módulos específicos, garantindo que a camada de aplicação permanecesse independente do framework web.

Para a construção do \textbf{front-end}, foi escolhido o framework \textbf{Next.js}, que é baseado em \textbf{React}. A escolha do Next.js foi estratégica devido aos seus recursos de Renderização no Lado do Servidor (SSR) e Geração de Site Estático (SSG). Tais funcionalidades são cruciais para uma plataforma de conteúdo como o FromCaio, pois garantem um excelente desempenho de carregamento das páginas e otimização para motores de busca (SEO), tornando o conteúdo facilmente encontrável.

A \textbf{modelagem de dados} foi realizada com a criação de um diagrama entidade-relacionamento (DER) para estruturar as principais entidades do sistema, como `Usuários`, `Artigos`, `Cursos` e `Categorias`. O banco de dados foi projetado para garantir a integridade referencial e suportar de forma eficiente as consultas necessárias para a exibição dos conteúdos e gerenciamento da plataforma.

\section{Implementação e Validação do Sistema}

A fase de implementação seguiu uma abordagem iterativa, desenvolvendo e integrando as funcionalidades em ciclos. O desenvolvimento foi organizado da seguinte forma:

\begin{enumerate}
    \item \textbf{Desenvolvimento do Back-end (API):} Inicialmente, foi implementada a API com todos os \textit{endpoints} necessários para o CRUD (Create, Read, Update, Delete) das entidades, além das regras de negócio para permissões, publicação e processamento do conteúdo.
    \item \textbf{Desenvolvimento do Front-end:} Com a API funcional, a interface do usuário foi construída, consumindo os dados do \textit{back-end}. Cada componente da interface, como o editor de texto, a visualização de artigos e os painéis de usuário, foi desenvolvido de forma modular.
    \item \textbf{Validação:} Ao final de cada ciclo de implementação, foram realizados testes para validar as funcionalidades. Foram aplicados testes unitários nas regras de negócio críticas do \textit{back-end} e testes de integração para verificar a comunicação entre o \textit{front-end} e a API. Adicionalmente, foi realizada uma validação funcional manual, simulando o uso da plataforma por autores e leitores para garantir que os requisitos foram atendidos e que a experiência de uso era intuitiva e fluida.
\end{enumerate}

\section{Cronograma}

O cronograma a seguir apresenta as etapas planejadas para o desenvolvimento do projeto \textit{FromCaio}, abrangendo desde o planejamento inicial até a finalização e entrega da monografia.

\begin{table}[h!]
\centering
\caption{Cronograma de Desenvolvimento do Projeto FromCaio}
\label{tab:cronograma}
\renewcommand{\arraystretch}{1.3}
\begin{tabular}{@{}p{3.2cm} p{5.5cm} p{2.5cm} p{2.5cm}@{}}
\toprule
\textbf{Etapa} & \textbf{Atividade Específica} & \textbf{Início} & \textbf{Término} \\
\midrule

\multirow{3}{*}{\parbox[l]{3cm}{Planejamento\\e Arquitetura}}
& Detalhamento dos requisitos & 21/07/2025 & 27/07/2025 \\
& Definição da arquitetura e tecnologias & 21/07/2025 & 27/07/2025 \\
& Modelagem do banco de dados (DER) & 21/07/2025 & 27/07/2025 \\
\midrule

\multirow{4}{*}{\parbox[l]{3cm}{Desenvolvimento do Back-End}}
& Configuração do ambiente (API) & 28/07/2025 & 03/08/2025 \\
& Sistema de autenticação de usuários & 28/07/2025 & 03/08/2025 \\
& CRUD para Artigos e Cursos & 04/08/2025 & 10/08/2025 \\
& Lógica de processamento de conteúdo & 04/08/2025 & 10/08/2025 \\
\midrule

\multirow{4}{*}{\parbox[l]{3cm}{Desenvolvimento do Front-End}}
& Configuração do ambiente (UI) & 11/08/2025 & 17/08/2025 \\
& Páginas estáticas e painel do usuário & 11/08/2025 & 24/08/2025 \\
& Implementação do editor de texto & 18/08/2025 & 24/08/2025 \\
& Telas de visualização de conteúdo & 25/08/2025 & 31/08/2025 \\
\midrule

\multirow{3}{*}{\parbox[l]{3cm}{Integração e Testes}}
& Conexão do Front-end com a API & 25/08/2025 & 31/08/2025 \\
& Testes de ponta a ponta (E2E) & 01/09/2025 & 07/09/2025 \\
& Correção de bugs e refatoração & 01/09/2025 & 07/09/2025 \\
\midrule

\multirow{3}{*}{\parbox[l]{3cm}{Finalização e Documentação}}
& Validação final e ajustes & 08/09/2025 & 14/11/2025 \\
& Redação da monografia & 08/09/2025 & 14/11/2025 \\
& Preparação da apresentação & 08/09/2025 & 14/11/2025 \\
\bottomrule
\end{tabular}
\end{table}