\chapter{Metodologia}\label{cap:metodologia}

O desenvolvimento da plataforma \textit{GranaLivre} segue uma abordagem iterativa e incremental, organizada em quatro \textit{sprints} com duração aproximada de três semanas cada. Essa estrutura busca garantir entregas contínuas, com ciclos curtos de planejamento, implementação e validação. O cronograma definido contempla as seguintes entregas:

\begin{itemize}
    \item \textbf{Sprint 1}: 23/09/2025
    \item \textbf{Sprint 2}: 21/10/2025
    \item \textbf{Sprint 3}: 11/11/2025
    \item \textbf{Sprint 4}: 09/12/2025
\end{itemize}

\section{Sprint 1: Organização e Estruturação}

O primeiro ciclo concentrou-se na preparação do ambiente e na definição de diretrizes para todo o projeto. As principais atividades realizadas foram:

\begin{itemize}
    \item Estruturação da documentação inicial.
    \item Organização do repositório no GitHub, incluindo definição de \textit{branching model} e fluxo de trabalho.
    \item Definição de papéis da equipe.
    \item Elaboração de um diagrama de casos de uso.
    \item Modelagem do banco de dados relacional, com diagrama entidade-relacionamento (DER).
    \item Criação de protótipos de baixa fidelidade de todas as telas.
    \item Estabelecimento do cronograma de entregas para os quatro \textit{sprints}.
\end{itemize}

Este conjunto de entregas garantiu uma base sólida, tanto técnica quanto organizacional, para o desenvolvimento nas etapas seguintes.

\section{Sprints 2 a 4: Desenvolvimento Iterativo}

A partir da segunda \textit{sprint}, o foco \textbf{será} o desenvolvimento integrado das funcionalidades. Cada funcionalidade \textbf{será tratada} como um todo, contemplando:

\begin{itemize}
    \item \textbf{Front-end:} implementação das interfaces utilizando \textbf{Next.js}, com HTML, CSS e JavaScript.
    \item \textbf{Back-end:} desenvolvimento de serviços e regras de negócio com \textbf{Django} (Python).
    \item \textbf{Banco de dados:} modelagem e integração com o \textbf{PostgreSQL}.
    \item \textbf{Containerização:} uso de \textbf{Docker} para padronizar o ambiente de desenvolvimento e facilitar a implantação.
\end{itemize}

Esse modelo permitirá que funcionalidades completas, como cadastro e autenticação de usuários, sejam projetadas, implementadas e testadas em uma mesma \textit{sprint}, assegurando sua validação prática desde o início.

\section{Gestão de Qualidade de Código}

Além da divisão funcional, a equipe estabeleceu um papel específico de revisão de código. Um dos desenvolvedores será responsável por avaliar a qualidade da implementação, garantindo a aderência a boas práticas de escrita e a consistência do estilo. Nenhum código será integrado ao repositório principal sem a devida aprovação via \textit{pull request}, assegurando a qualidade contínua do projeto.

\section{Cronograma}

O cronograma de execução do projeto está detalhado na Tabela \ref{tab:cronograma}. Ele foi estruturado de forma a permitir a progressão gradual das entregas, conciliando documentação, prototipação e desenvolvimento.

\begin{table}[h!]
\centering
\caption{Cronograma de Desenvolvimento do Projeto GranaLivre}
\label{tab:cronograma}
\renewcommand{\arraystretch}{1.3}
\begin{tabular}{@{}p{3.2cm} p{7cm} p{2.5cm}@{}}
\toprule
\textbf{Sprint} & \textbf{Atividades Principais} & \textbf{Entrega} \\
\midrule
Sprint 1 & Documentação, organização do repositório, definição de papéis, casos de uso, DER e protótipos & 23/09/2025 \\
Sprint 2 & Desenvolvimento inicial (cadastro, login e autenticação); integração front/back/banco & 21/10/2025 \\
Sprint 3 & Expansão de funcionalidades (entradas, saídas, resumo, entradas contas); testes e ajustes & 11/11/2025 \\
Sprint 4 & Expansão de funcionalidades (despesas recorrentes, investimentos e patrimônio); testes e ajustes & 09/12/2025 \\
\bottomrule
\end{tabular}
\end{table}